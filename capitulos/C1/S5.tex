\section{Sucesiones exactas}

Decimos que una sucesión de $A$-módulos y $A$-homomorfismos
\begin{equation}\label{eq:1.1}%
  \cdots \to M_{i-1}
    \labto{f_i} M_i 
    \labto{f_{i+1}} M_{i+1}
    \to \cdots  
\end{equation}
es \emph{exacta} en $M_i$ si $\im f_i = \ker f_{i+1}$. La sucesión es \emph{exacta} si es exacta en cada $M_i$. En particular:
\begin{enumerate}
  \item $0 \to M' \labto{f} M $ es exacta si y solo si $f$ es inyectiva.
  \item $M \labto{g} M'' \to 0$ es exacta si y solo si $g$ es suprayectiva.
  \item $0 \to M' \labto{f} M \labto{g} M'' \to 0$ es exacta si y solo si $f$ es inyectiva, $g$ es suprayectiva y $g$ induce un isomorfismo de $\coker f = M / \im f$ sobre $M''$.
\end{enumerate}

Una sucesión del tipo $(c)$ es llamada una \emph{sucesión exacta corta}- Cualquier sucesión exacta larga \eqref{eq:1.1} se puede dividir en sucesiones exactas cortas: si $N_i = \im f_i = \ker f_{i+1}$, tenemos para cada $i$ sucesiones exactas cortas $0 \to N_i \to M_i \to N_{i+1} \to 0$.

\begin{lemma}[de la serpiente]\label{prop:1.5.1}%
  Sea 
  \[
    \begin{tikzcd}
      0 \arrow[r] & M' \arrow[r, "f"] \arrow[d, "h'"] & M \arrow[r, "g"] \arrow[d, "h"] & M'' \arrow[r] \arrow[d, "h''"] & 0 \\
      0 \arrow[r] & N' \arrow[r, "f'"']               & N \arrow[r, "g'"']              & N'' \arrow[r]                  & 0
    \end{tikzcd}
  \]
  un diagrama conmutativo de $A$-módulos y homomorfismos, con los renglones exactos. Entonces existe una sucesión exacta
  \begin{equation}\label{eq:1.5}%
    0 \to \ker h' \labto{\bar f} \ker h \labto{\bar g} \ker h'' \labto{\delta} 
      \coker h' \labto{\bar f'} \coker h \labto{\bar g'} \coker h'' \to 0.
  \end{equation}
  en la que $\bar f$, $\bar g$ son restricciones de $f$, $g$ y $\bar f'$, $\bar g'$ son inducidas por $\bar f'$, $\bar g'$ son inducidas por $f'$, $g'$. \qed
\end{lemma}

El \emph{homomorfismo frontera $\delta$} se define como sigue: si $x'' \in \ker h''$, tenemos que $x'' = g(x)$ para algún $x \in M$ y $g'(h(x)) = h''(g(x)) = 0$, de aquí $h(x) \in \ker g' = \im f'$, así que $h(x) = f'(y')$ para algún $y' \in N'$. Entonces $\delta(x'')$ se define como la imagen de $y'$ en el $\coker h'$. La verificación de que $\delta$ está bien definido y de que la sucesión \eqref{eq:1.5} es exacta es un ejercicio directo en persecución de elementos en el diagrama, que dejamos al lector.



\ExerciseSection

\begin{ExerciseList}
  \item Sea $0 \to M' \labto{f} M \labto{g} M'' \to 0$ una sucesión exacta de $A$-módulos. Muestre que $M$ es noetheriano si y solo si $M'$ y $M''$ son noetherianos.
  \item Pruebe en detalle la proposición \ref{prop:1.5.1}.
\end{ExerciseList}