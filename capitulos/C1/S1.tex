\section{Definiciones básicas}

Todos los anillos serán conmutativos y tendrán un elemento identidad; un homomorfismo de anillos $\varphi\colon A \to B$ satisface $\varphi(1_A) = 1_B$. Un anillo $B$ junto con un homomorfismo de anillos $A\to B$ será una \emph{$A$-álgebra}. Usamos esta terminología principalmente cuando $A$ es un subanillo de $B$. En este caso, para elementos $\beta_1, \ldots, \beta_m$ de $B$ denotaremos como $A[\beta_1,\ldots,\beta_m]$ al subanillo más pequeño de $B$ conteniendo $A$ y los $\beta_i$. Consiste de los polinomios en los $\beta_i$ con coeficientes en $A$, i.e., elementos de la forma
\[
  \sum a_{i_1\cdots i_m} \beta_1^{i_1}\cdots \beta_m^{i_m}, \qquad  a_{i_1\cdots i_m} \in A.
\]

Nos referimos también a $A[\beta_1,\ldots, \beta_m]$ como al $A$-subálgebra de $B$ \emph{generada} por los $\beta_i$ y cuando $B = A[\beta_1,\ldots, \beta_m]$ decimos que los $\beta_i$ \emph{generan} $B$ como una $A$-álgebra.