\chapter{Preliminares de álgebra conmutativa}




\section{Definiciones básicas}

Todos los anillos serán conmutativos y tendrán un elemento identidad; un homomorfismo de anillos $\varphi\colon A \to B$ satisface $\varphi(1_A) = !_B$. Un anillo $B$ junto con un homomorfismo de anillos $A\to B$ será una \emph{$A$-álgebra}. Usamos esta terminología principalmente cuando $A$ es un subanillo de $B$. En este caso, para elementos $\beta_1, \ldots, \beta_m$ de $B$, $A[\beta_1,\ldots,\beta_m]$ denota el subanillo más pequeño de $B$ conteniendo $A$ y los $\beta_i$. Consiste de los polinomios en los $\beta_i$ con coeficientes en $A$, i.e., elementos de la forma
\[
  \sum a_{i_1\cdots i_m} \beta_1^{i_1}\cdots \beta_m^{i_m}, \qquad  a_{i_1\cdots i_m} \in A.
\]

Nos referimos también a $A[\beta_1,\ldots, \beta_m]$ como al $A$-subálgebra de $B$ \emph{generada} por los $\beta_i$ y cuando $B = A[\beta_1,\ldots, \beta_m]$ decimos que los $\beta_i$ \emph{generan} $B$ como una $A$-álgebra.




\section{Extensión y contracción}

Sea $f\colon A \to B$ un homomorfismo de anillos. Si $I$ es un ideal de $A$, el conjunto $f(I)$  no es necesariamente un ideal de $B$ (e.g. sea $f$ la inmersión de $\Z$ en $\Q$, el campo de los racionales y sea $I$ cualquier ideal distinto de cero de $\Z$). Definimos la \emph{extensión} $I^\ext$ de $I$ como el ideal $f(I)B$ generado por el conjunto $f(I)$ en $B$. Más explícitamente $I^\ext$ es el conjunto de las sumas finitas $\sum y_i f(x_i)$ donde $x_i \in I$ e $y_i \in B$.

Si $J$ es un ideal de $B$, entonces $f^{-1}(J)$ es siempre un ideal de $A$, llamado la \emph{contracción}  $J^\cont$ de $J$. Si $J$ es primo, entonces $J^\cont$ es primo. Si $I$ es primo $I^\ext$ no es necesariamente primo (por ejemplo $f\colon \Z \to \Q$ con $I\neq 0$; entonces $I^\ext = \Q$, que no es un ideal primo).

\begin{proposition}
  Sea $f\colon A \to B$ un homomorfismo de anillos, $I\subseteq A$ y $J\subseteq B$ ideales, se cumple las siguiente propiedades:
  \begin{enumerate}
    \item $I \subseteq I^{\ext\cont}$ y $J^{\cont\ext} \subseteq J$.
    \item $J^\cont = J^{\cont\ext\cont}$ y $I^\ext = I^{\ext\cont\ext}$.
  \end{enumerate}
\end{proposition}
\begin{proof}
  $(a)$ es trivial y $(b)$ se sigue de $(a)$.
\end{proof}




\ExerciseSection

\begin{exerciseList}
  \item Si $I_1$, $I_2$ son ideales de $A$ y $J_1$, $J_2$ son ideales de $B$, entonces
    \begin{enumerate}
      \item $(I_1+I_2)^\ext = I_1^\ext + I_2^\ext$ y $(J_1+J_2)^\cont \supseteq J_1^\cont + J_2^\cont$.
      \item $(I_1 \cap I_2)^\ext \subseteq I_1^\ext \cap I_2^\ext$ y $(J_1 \cap J_2)^\cont = J_1^\cont \cap J_2^\cont$.
      \item $(I_1 I_2)^\ext = I_1^\ext I_2^\ext$ y $(J_1 J_2)^\cont \supseteq J_1^\cont J_2^\cont$
    \end{enumerate}

  \item Suponga que $f\colon A \to B $ y $g\colon A \to C$ son homomorfismos de anillos y que $h\colon B \to C$ es un isomorfismo que satisface $g = h \circ f$. Muestr que para cualquier ideal $I$ de $A$, $h$ se restringe a un isomorfismo entre $f(I)B$ y $g(I)C$.
\end{exerciseList}




\section{Anillos noetherianos}

\begin{lemma}
  Las condiciones siguientes sobre un anillo son equivalentes:
  \begin{enumerate}
    \item Cada ideal en $A$ es finitamente generado.
    \item Cada cadena ascendente de ideales
      \[ I_1 \subseteq I_2 \subseteq \cdots \subseteq I_n \cdots \]
      se estaciona, i.e., después de un cierto punto $I_n = I_{n+1} = \cdots$.
    \item Cada conjunto no vacío $S$ de ideales en $A$ tiene un elemento maximal $I$, i.e., existe un ideal $I$ en $S$ que no está contenido en cualquier otro ideal en $S$.
  \end{enumerate}
\end{lemma}
\begin{proof}
  $(a) \Rightarrow (b)$ Sea $ I = \bigcup I_i$, es un ideal y de aquí es finitamente generado, digamos $I = (a_1,\ldots, a_r)$. Para algún $n$ el ideal $I_n$ contendrá a todos los $a_i$ y de este modo $I_n = I_{n+1} = \cdots$.

  \nextpart
  $(b) \Rightarrow (a)$ Considere un ideal $I$. Si $I = (0)$, entonces $I$ es generado por el conjunto vacío, el cual es finito. De otro modo, existe un elemento $a_1 \in I$, $a_1 \neq 0$. Si $I = (a_1)$ entonces, ciertamente, $I$ es finitamente generado. Si nom existe un elemento $a_2 \in I$ tal que $(a_1) \subsetneq (a_1, a_2)$. Continuando de esta manera, obtenemos una cadena de ideales
  \[
    (a_1) \subsetneq (a_1, a_2) \subsetneq (a_1,a_2,a_3) \subsetneq \cdots
  \]
  Este proceso debe detenerse en algún momento con $(a_1,\ldots,a_n) = I$.

  \nextpart
  $(b)\Rightarrow (c)$ Sea $I_1 \in S$. Si $I_1$ no es un elemento maximal de $S$, entonces existe un $I_2$ tal que $I_1 \subsetneq I_2$. Si $I_2$ no es maximal, existe un $I_3$, etc. De $(b)$, sabemos que este proceso conducirá a un elemento maximal después de, únicamente, un número finito de pasos.

  \nextpart
  $(c)\Rightarrow (b)$ Aplique $(c)$ al conjunto $S = \{I_1,I_2,\ldots\}$.
\end{proof}

Un anillo satisfaciendo cualquiera de las condiciones equivalentes es llamado \emph{noetheriano}.

Un teorema famoso, el \emph{teorema de la base de Hilbert}, establece que el anillo no polinomios $k[x_1,\ldots,x_n]$ es noetheriano. En la práctica casi todos los anillos que surgen naturalmente en la \emph{Teoría de Números Algebraicos} o en \emph{Geometría Algebraica} son noetherianos; pero, desde luego, no todos los anillos son noetherianos.

Por ejemplo, $k[x_1,\ldots,x_n,\ldots]$ no es noetheriano: $x_1, \ldots, x_n$ es un conjunto mínimo de generadores para el ideal $(x_1,\ldots,x_n)$ en $k[x_1,\ldots,x_n]$ y $x_1,\ldots,x_n,\ldots$ es un conjunto mínimo de generadores de $(x_1,\ldots,x_n,\ldots)$ en $k[x_1,\ldots,x_n,\ldots]$.

\begin{proposition}
  Cada elemento distinto de cero que no es una unidad de un dominio entero noetheriano se puede escribir como un producto de elementos irreducibles.
\end{proposition}
\begin{proof}
  Suponga que el enunciado es falso y escoja un elemento $a\in A$ que contradiga el enunciado y es tal que $(a)$ es maximal entre los ideales generados por dichos elementos (aquí usamos que $A$ es noetheriano). Ya que $a$ no puede ser escrito como un producto de elementos irreducibles, él mismo no es irreducible y, de este modo, $a = bc$ con $b$ y $c$ no unidades. Claramente $(a) \subseteq (b) $ y los ideales no pueden ser iguales por que de lo contrario $c$ sería unidad. De la maximalidad de $a$, podemos deducir que $b$ puede ser escrito como un producto de elementos irreducibles y similarmente para $c$. Por lo tanto, $a$ es un producto de elementos irreducible (contradicción).
\end{proof}