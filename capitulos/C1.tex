\chapter{Preliminares de álgebra conmutativa}




\section{Definiciones básicas}

Todos los anillos serán conmutativos y tendrán un elemento identidad; un homomorfismo de anillos $\varphi\colon A \to B$ satisface $\varphi(1_A) = !_B$. Un anillo $B$ junto con un homomorfismo de anillos $A\to B$ será una \emph{$A$-álgebra}. Usamos esta terminología principalmente cuando $A$ es un subanillo de $B$. En este caso, para elementos $\beta_1, \ldots, \beta_m$ de $B$, $A[\beta_1,\ldots,\beta_m]$ denota el subanillo más pequeño de $B$ conteniendo $A$ y los $\beta_i$. Consiste de los polinomios en los $\beta_i$ con coeficientes en $A$, i.e., elementos de la forma
\[
  \sum a_{i_1\cdots i_m} \beta_1^{i_1}\cdots \beta_m^{i_m}, \qquad  a_{i_1\cdots i_m} \in A.
\]

Nos referimos también a $A[\beta_1,\ldots, \beta_m]$ como al $A$-subálgebra de $B$ \emph{generada} por los $\beta_i$ y cuando $B = A[\beta_1,\ldots, \beta_m]$ decimos que los $\beta_i$ \emph{generan} $B$ como una $A$-álgebra.




\section{Extensión y contracción}

Sea $f\colon A \to B$ un homomorfismo de anillos. Si $I$ es un ideal de $A$, el conjunto $f(I)$  no es necesariamente un ideal de $B$ (e.g. sea $f$ la inmersión de $\Z$ en $\Q$, el campo de los racionales y sea $I$ cualquier ideal distinto de cero de $\Z$). Definimos la \emph{extensión} $I^\ext$ de $I$ como el ideal $f(I)B$ generado por el conjunto $f(I)$ en $B$. Más explícitamente $I^\ext$ es el conjunto de las sumas finitas $\sum y_i f(x_i)$ donde $x_i \in I$ e $y_i \in B$.

Si $J$ es un ideal de $B$, entonces $f^{-1}(J)$ es siempre un ideal de $A$, llamado la \emph{contracción}  $J^\cont$ de $J$. Si $J$ es primo, entonces $J^\cont$ es primo. Si $I$ es primo $I^\ext$ no es necesariamente primo (por ejemplo $f\colon \Z \to \Q$ con $I\neq 0$; entonces $I^\ext = \Q$, que no es un ideal primo).

\begin{proposition}
  Sea $f\colon A \to B$ un homomorfismo de anillos, $I\subseteq A$ y $J\subseteq B$ ideales, se cumple las siguiente propiedades:
  \begin{enumerate}
    \item $I \subseteq I^{\ext\cont}$ y $J^{\cont\ext} \subseteq J$.
    \item $J^\cont = J^{\cont\ext\cont}$ y $I^\ext = I^{\ext\cont\ext}$.
  \end{enumerate}
\end{proposition}
\begin{proof}
  $(a)$ es trivial y $(b)$ se sigue de $(a)$.
\end{proof}




\ExerciseSection

\begin{exerciseList}
  \item Si $I_1$, $I_2$ son ideales de $A$ y $J_1$, $J_2$ son ideales de $B$, entonces
    \begin{enumerate}
      \item $(I_1+I_2)^\ext = I_1^\ext + I_2^\ext$ y $(J_1+J_2)^\cont \supseteq J_1^\cont + J_2^\cont$.
      \item $(I_1 \cap I_2)^\ext \subseteq I_1^\ext \cap I_2^\ext$ y $(J_1 \cap J_2)^\cont = J_1^\cont \cap J_2^\cont$.
      \item $(I_1 I_2)^\ext = I_1^\ext I_2^\ext$ y $(J_1 J_2)^\cont \supseteq J_1^\cont J_2^\cont$
    \end{enumerate}

  \item Suponga que $f\colon A \to B $ y $g\colon A \to C$ son homomorfismos de anillos y que $h\colon B \to C$ es un isomorfismo que satisface $g = h \circ f$. Muestra que para cualquier ideal $I$ de $A$, $h$ se restringe a un isomorfismo entre $f(I)B$ y $g(I)C$.
\end{exerciseList}




\section{Anillos noetherianos}

\begin{lemma}
  Las condiciones siguientes sobre un anillo son equivalentes:
  \begin{enumerate}
    \item Cada ideal en $A$ es finitamente generado.
    \item Cada cadena ascendente de ideales
      \[ I_1 \subseteq I_2 \subseteq \cdots \subseteq I_n \cdots \]
      se estaciona, i.e., después de un cierto punto $I_n = I_{n+1} = \cdots$.
    \item Cada conjunto no vacío $S$ de ideales en $A$ tiene un elemento maximal $I$, i.e., existe un ideal $I$ en $S$ que no está contenido en cualquier otro ideal en $S$.
  \end{enumerate}
\end{lemma}
\begin{proof}
  $(a) \Rightarrow (b)$ Sea $ I = \bigcup I_i$, es un ideal y de aquí es finitamente generado, digamos $I = (a_1,\ldots, a_r)$. Para algún $n$ el ideal $I_n$ contendrá a todos los $a_i$ y de este modo $I_n = I_{n+1} = \cdots$.

  \nextpart
  $(b) \Rightarrow (a)$ Considere un ideal $I$. Si $I = (0)$, entonces $I$ es generado por el conjunto vacío, el cual es finito. De otro modo, existe un elemento $a_1 \in I$, $a_1 \neq 0$. Si $I = (a_1)$ entonces, ciertamente, $I$ es finitamente generado. Si nom existe un elemento $a_2 \in I$ tal que $(a_1) \subsetneq (a_1, a_2)$. Continuando de esta manera, obtenemos una cadena de ideales
  \[
    (a_1) \subsetneq (a_1, a_2) \subsetneq (a_1,a_2,a_3) \subsetneq \cdots
  \]
  Este proceso debe detenerse en algún momento con $(a_1,\ldots,a_n) = I$.

  \nextpart
  $(b)\Rightarrow (c)$ Sea $I_1 \in S$. Si $I_1$ no es un elemento maximal de $S$, entonces existe un $I_2$ tal que $I_1 \subsetneq I_2$. Si $I_2$ no es maximal, existe un $I_3$, etc. De $(b)$, sabemos que este proceso conducirá a un elemento maximal después de, únicamente, un número finito de pasos.

  \nextpart
  $(c)\Rightarrow (b)$ Aplique $(c)$ al conjunto $S = \{I_1,I_2,\ldots\}$.
\end{proof}

Un anillo satisfaciendo cualquiera de las condiciones equivalentes es llamado \emph{noetheriano}.

Un teorema famoso, el \emph{teorema de la base de Hilbert}, establece que el anillo no polinomios $k[x_1,\ldots,x_n]$ es noetheriano. En la práctica casi todos los anillos que surgen naturalmente en la \emph{Teoría de Números Algebraicos} o en \emph{Geometría Algebraica} son noetherianos; pero, desde luego, no todos los anillos son noetherianos.

Por ejemplo, $k[x_1,\ldots,x_n,\ldots]$ no es noetheriano: $x_1, \ldots, x_n$ es un conjunto mínimo de generadores para el ideal $(x_1,\ldots,x_n)$ en $k[x_1,\ldots,x_n]$ y $x_1,\ldots,x_n,\ldots$ es un conjunto mínimo de generadores de $(x_1,\ldots,x_n,\ldots)$ en $k[x_1,\ldots,x_n,\ldots]$.

\begin{proposition}
  Cada elemento distinto de cero que no es una unidad de un dominio entero noetheriano se puede escribir como un producto de elementos irreducibles.
\end{proposition}
\begin{proof}
  Suponga que el enunciado es falso y escoja un elemento $a\in A$ que contradiga el enunciado y es tal que $(a)$ es maximal entre los ideales generados por dichos elementos (aquí usamos que $A$ es noetheriano). Ya que $a$ no puede ser escrito como un producto de elementos irreducibles, él mismo no es irreducible y, de este modo, $a = bc$ con $b$ y $c$ no unidades. Claramente $(a) \subseteq (b) $ y los ideales no pueden ser iguales por que de lo contrario $c$ sería unidad. De la maximalidad de $a$, podemos deducir que $b$ puede ser escrito como un producto de elementos irreducibles y similarmente para $c$. Por lo tanto, $a$ es un producto de elementos irreducible (contradicción).
\end{proof}




\section{Módulos}

Sea $A$ un anillo. Un \emph{$A$-módulo} es un grupo abeliano $M$ junto con una función (multiplicación escalar) de $A\times M \to M$, $(a,x) \mapsto ax$ tal que se satisfacen los siguientes axiomas:
\begin{align*}
  a(x+y) &= ax + ay, \\
  (a+b)x &= ax + bx, \\
  (ab)x &= a(bx), \\
  1x &= x,
\end{align*}
para todo $a,b \in A$ y todo $x,y \in M$.

La noción de módulo es una generalización de varios conceptos familiares. Por ejemplo, un ideal $I$ de $A$ es un $A$-módulo. Si $A = k$ es un campo, entonces un $A$-módulo es un $k$-espacio vectorial. Un $\Z$ módulo es un grupo abeliano.

Un \emph{submódulo} $N$ de $M$ es un subgrupo de $M$ que es cerrado respecto a la multiplicación por elementos de $A$. El grupo abeliano $M/N$ hereda de $M$ una estructura de $A$-módulo definida por $a(x+N) = ax + N$. El $A$-módulo $M/N$ es el \emph{cociente} de $M$ por $N$.

La mayoría de las operaciones con ideales tienen sus análogas para módulos. Sea $M$ un $A$-módulo y sea $(M_i)_{i\in I}$ una familia de submódulos de $M$. Su \emph{suma} $\sum M_i$ es el conjunto de todas las sumas $\sum x_i$ donde $x_i \in M_i$ donde $x_i \in M_i$ para todo $i \in I$ y donde casi todas las $x_i$ son cero. El \emph{producto} $IM$, donde $I$ es un ideal de $A$ y $M$ un $A$-módulo, es el conjunto de todas las sumas finitas $\sum a_i x_i$ con $a_i \in I$ y $x_i \in M$, es un submódulo de $M$.

Si $x$ es un elemento de $M$, el conjunto de todos los múltiplos $ax$ con $a \in A$ es un submódulo de $M$, llamado el submódulo \emph{cíclico} generado por $x$ y se denota $Ax$ o $(x)$. Si $M = \sum_{i \in I} A x_i$, las $x_i$ se dicen que son un \emph{conjunto de generadores} de $M$. Un $A$-módulo $M$ se dice que es \emph{finitamente generado} si tiene un conjunto finito de generadores.

\begin{definition}
  Sean $M$ y $N$, $A$-módulos. Una función $f\colon M \to N $ es un homomorfismo de $A$-módulos (o es $A$-lineal) si 
    \[
      f(am+bn) = af(m) + bf(n)
    \]
  para todo $a,b \in A$ y $m,n \in M$.
\end{definition}

Tenemos las siguiente definiciones: 
\begin{itemize}
  \item Un \emph{endomorfismo} es un homomorfismo de $M$ a $M$.
  \item Un monomorfismo es un homomorfismo inyectivo.
  \item Un epimorfismo es un homomorfismo suprayectivo.
  \item Un isomorfismo es un homomorfismo biyectivo.
\end{itemize}

Si $f$ es un homomorfismo entonces el núcleo de $f$,
  \[
    \ker f = \{m \in M : f(m) = 0\}
  \]
y la imagen de $f$,
  \[
    \im f = \{f(m) : m \in M\}
  \]
son submódulos de $M$ y $N$, respectivamente. Si $A$ es un campo, un homomorfismo de $A$-módulos es una transformación lineal de espacios vectoriales. El conjunto de todos los homomorfismos de $A$-módulos de $M$ a $N$ es también un $A$-módulo con la suma de homomorfismos la suma usual de funciones, $(f+g)(m) = f(m) + g(m)$ y $(af)(m) = af(m)$ para toda $m \in M$. Este $A$-módulo se denota por $\Hom_A(M,N)$.

Todos los teoremas de isomorfismos se sostienen para $R$-módulos. Las pruebas invocan el teorema correspondiente para grupos y luego se prueba que los homomorfismos de grupos son también homomorfismos de $R$-módulos.

\emph{Primer teorema de isomorfismo}: Sean $M,N$, $A$-módulos y sea $f\colon M \to N$ un homomorfismo de $A$-módulos. Entonces $\ker f$ es un submódulo de $M$ tal que $M/\ker f \cong \im f$.

\emph{Segundo teorema de isomorfismo}: Sean $N,P$ submódulos del $A$-módulo $M$. Entonces $(N+P)/P \cong N/(N \cap P)$.

\emph{Tercer teorema de isomorfismo}: Sea $M$ un $A$-módulo y sean $N$ y $P$ submódulos de $M$ con $N \subseteq P$. Entonces $(M/N)/(P/N) \cong M/P$.

\emph{Teorema de correspondencia}: Sea $N$ un submódulo del $A$-módulo $M$. Existe una biyección entre los submódulos de $M$ que contienen $N$ y los submódulos de $M/N$. La correspondencia es dada por $P \leftrightarrow P/N$, para toda $P \supseteq N$. Esta correspondencia conmuta con el proceso de tomar sumas e intersecciones.

\begin{definition}
  Un $A$-módulo $M$ es \emph{noetheriano} si cada submódulo de $M$ es finitamente generado.
\end{definition}

Esto es equivalente a la condición de que $M$ satisface la condición de que $M$ satisface la condición de cadena ascendente sobre submódulos o que cualquier colección no vacía de submódulos de $M$ tiene un elemento maximal.

La importancia de los módulos noetherianos viene de la observación siguiente:

\begin{theorem}[de la base de Hilbert para módulos]
  Si $A$ es un anillo noetheriano y $M$ es un $A$-módulo finitamente generado, entonces $M$ es noetheriano.
\end{theorem}
\begin{proof}
  Suponga que $M$ es generado por $f_1, \ldots, f_t$ y sea $N$ un submódulo. Mostramos que $N$ es finitamente generado por inducción sobre $t$.

  Si $t = 1$, entonces el homomorfismo $A \to M$ mandando $1$ a $f_1$ es suprayectivo. La preimagen de $N$ es un ideal que es finitamente generado ya que $A$ es noetheriano. Las imágenes de sus generadores generan $N$.

  Ahora suponga que $t>1$. La imagen de $N$, $\overline N$, en $M/Af_1$ es finitamente generada por inducción. Sean $g_1,\ldots,g_s$ elementos de $N$ cuyas imágenes generan $\overline N$. Ya que $Af_1 \subseteq M$ es generado por un elemento, su submódulo $N \cap Af_1$ es finitamente generado, digamos por $h_1,\ldots,h_r$.

  Mostraremos que los elementos $h_1,\ldots,h_r$ y $g_1,\ldots,g_s$ juntos generan $N$. Dado $n \in N$, la imagen de $n$ en $\overline N$ es una combinación lineal de las imágenes de las $g_i$; restando las combinaciones lineales correspondientes de las $g_i$ de $N$, obtenemos un elemento de $N \cap Af_1$, que es una combinación lineal de las $h_i$ por hipótesis. Esto muestra que $n$ es una combinación lineal de las $g_i$ y $h_i$.
\end{proof}

\begin{lemma}[Nakayama]
  Sea $A$ un anillo local y sea $I$ un ideal propio de $A$. Sea $M$ un $A$-módulo finitamente generado.
  \begin{enumerate}
    \item Si $IM = M$, entonces $M = 0$.
    \item Si $N$ es un submódulo de $M$ tal que $N+IM = M$, entonces $N = M$.
  \end{enumerate}
\end{lemma}
\begin{proof}
  $(a)$ Suponga que $M \neq 0$. Entre todos los conjuntos posibles de generadores para $M$, escoja uno $\{m_1,\ldots,m_k\}$ tomando el menor número posible de elementos. Por hipótesis podemos escribir
  \[
    m_k = a_1 m_1 + a_2 m_2 + \cdots + a_k m_k
      \quad\text{para ciertos}\quad a_i \in I.
  \]
  Entonces
  \[
    (1-a_k)m_k = a_1 m_1 + a_2 m_2 + \cdots + a_{k-1} m_{k-1}.
  \]
  Si $J$ es el ideal maximal de $A$, $I \subseteq J$ y $1-a_k \notin J$. De aquí, $1-a_k$ es una unidad y por lo tanto $\{m_1,\ldots,m_{k-1}\}$ genera $M$. Esto contradice nuestra selección de $\{m_1,\ldots,m_k\}$ y por lo tanto $M = 0$.

  (b) Mostraremos que $I(M/N) = M/N$ y entonces, aplicando la primera parte del lema, concluimos que $M/N = 0$. Considere $m+N$, $m \in M$. Por hipótesis, podemos escribir
  \[
    m + n + \sum a_i m_i,
      \qquad \text{con}\ a_i \in I,
      \qquad \text{con}\ m_i \in M.
  \]
  Por consiguiente,
  \[
    m+N = \sum a_i m_i + N = \sum a_i (m_i + N),
  \]
  de donde $m + N \in I(M/N)$.
\end{proof}

En el lema, la hipótesis de que $M$ sea finitamente generado es crucial. Por ejemplo, sea
\[
  \Z_{(5)} = \{q \in \Q : q = a/b \ \text{con}\ a,b \in \Z, 5 \nmid b\}.
\]
Entonces $q = a/b$ es una unidad de $\Z_{(5)}$ si y solo si $5 \nmid a$. Por lo tanto, las no unidades de $\Z_{(5)}$ son
\[
  \{ a/b \in \Z_{(5)} : 5 \mid a \} = 5\Z_{(5)}.
\]

Este es un ideal, así que $\Z_{(5)}$ es un anillo local con ideal maximal $5\Z_{(5)}$. Considere $\Q$ como un $\Z_{(5)}$-módulo. Entonces $\Q = 5\Z_{(5)}\Q$, pero $\Q \neq 0$. El punto es que $\Q$ no es finitamente generado sobre $\Z_{(5)}$,
\[
  \Q = \Z_{(5)} + \Z_{(5)} \frac{1}{5} + \Z_{(5)}\frac{1}{5^2} + \cdots .
\]




\section{Sucesiones exactas}

Decimos que una sucesión de $A$-módulos y $A$-homomorfismos
\begin{equation}\label{eq:1.1}%
  \cdots \to M_{i-1}
    \labto{f_i} M_i 
    \labto{f_{i+1}} M_{i+1}
    \to \cdots  
\end{equation}
es \emph{exacta} en $M_i$ si $\im f_i = \ker f_{i+1}$. La sucesión es \emph{exacta} si es exacta en cada $M_i$. En particular:
\begin{enumerate}
  \item $0 \to M' \labto{f} M $ es exacta si y solo si $f$ es inyectiva.
  \item $M \labto{g} M'' \to 0$ es exacta si y solo si $g$ es suprayectiva.
  \item $0 \to M' \labto{f} M \labto{g} M'' \to 0$ es exacta si y solo si $f$ es inyectiva, $g$ es suprayectiva y $g$ induce un isomorfismo de $\coker f = M / \im f$ sobre $M''$.
\end{enumerate}

Una sucesión del tipo $(c)$ es llamada una \emph{sucesión exacta corta}- Cualquier sucesión exacta larga \eqref{eq:1.1} se puede dividir en sucesiones exactas cortas: si $N_i = \im f_i = \ker f_{i+1}$, tenemos para cada $i$ sucesiones exactas cortas $0 \to N_i \to M_i \to N_{i+1} \to 0$.

\begin{lemma}[de la serpiente]\label{prop:1.5.1}%
  Sea 
  \[
    \begin{tikzcd}
      0 \arrow[r] & M' \arrow[r, "f"] \arrow[d, "h'"] & M \arrow[r, "g"] \arrow[d, "h"] & M'' \arrow[r] \arrow[d, "h''"] & 0 \\
      0 \arrow[r] & N' \arrow[r, "f'"']               & N \arrow[r, "g'"']              & N'' \arrow[r]                  & 0
    \end{tikzcd}
  \]
  un diagrama conmutativo de $A$-módulos y homomorfismos, con los renglones exactos. Entonces existe una sucesión exacta
  \begin{equation}\label{eq:1.5}%
    0 \to \ker h' \labto{\bar f} \ker h \labto{\bar g} \ker h'' \labto{\delta} 
      \coker h' \labto{\bar f'} \coker h \labto{\bar g'} \coker h'' \to 0.
  \end{equation}
  en la que $\bar f$, $\bar g$ son restricciones de $f$, $g$ y $\bar f'$, $\bar g'$ son inducidas por $\bar f'$, $\bar g'$ son inducidas por $f'$, $g'$. \qed
\end{lemma}

El \emph{homomorfismo frontera $\delta$} se define como sigue: si $x'' \in \ker h''$, tenemos que $x'' = g(x)$ para algún $x \in M$ y $g'(h(x)) = h''(g(x)) = 0$, de aquí $h(x) \in \ker g' = \im f'$, así que $h(x) = f'(y')$ para algún $y' \in N'$. Entonces $\delta(x'')$ se define como la imagen de $y'$ en el $\coker h'$. La verificación de que $\delta$ está bien definido y de que la sucesión \eqref{eq:1.5} es exacta es un ejercicio directo en persecución de elementos en el diagrama, que dejamos al lector.



\ExerciseSection

\begin{exerciseList}
  \item Sea $0 \to M' \labto{f} M \labto{g} M'' \to 0$ una sucesión exacta de $A$-módulos. Muestre que $M$ es noetheriano si y solo si $M'$ y $M''$ son noetherianos.
  \item Pruebe en detalle la proposición \ref{prop:1.5.1}.
\end{exerciseList}




\section{Anillo de fracciones}

See $A$ un dominio entero; existe un campo $K \supset A$, llamado el \emph{campo de fracciones} de $A$ con la propiedad de que cada $c \in K$ puede ser escrita en la forma $c = ab^{-1}$, $a, b \in A$, $b \neq 0$. Por ejemplo, $\Q$ es el campo de fracciones de $\Z$ y $k(x)$ es el campo de fracciones de $k[x]$.

Sea $A$ un dominio entero con campo de fracciones $K$. Un subconjunto $S$ de $A$ se llama \emph{multiplicativo} si $0 \notin S$, $1 \in S$ y $S$ es cerrado bajo la multiplicación. Si $S$ es un subconjunto multiplicativo, entonces Definimos
\[
  S^{-1}A = \set{\frac{a}{b} : b \in S}.
\]
Obviamente es un subanillo de $K$, llamado el \emph{anillo de fracciones de $A$ con respecto a $S$}. Tenemos también un homomorfismo inyectivo de anillos $f\colon A \to S^{-1}A$ definido por $f(x) = x/1$, que tiene la siguiente propiedad universal.

\begin{proposition}
  Sea $g\colon A \to B$ un homomorfismo de anillos tal que $g(s)$ es una unidad en $B$ para todo $s \in S$. Entonces existe un único homomorfismo $h\colon S^{-1}A \to B$ tal que $g = h \circ f$.
\end{proposition}
\begin{proof}
  Unicidad. Si $h$ satisface las condiciones, entonces $h(a/1) = h(f(a)) = g(a)$ para todo $a \in A$; de aquí, si $s \in S$, 
  \[
    h\paren{\frac{1}{s}} = h\paren{\paren{\frac{s}{1}}^{-1}} =  h\paren{\frac{s}{1}}^{-1} = g(s)^{-1}
  \]
  y por lo tanto $h(a/s) = h(a/1) h(1/s) = g(a)g(s)^{-1}$, así que $h$ está determinado de manera única por $g$.

  Existencia. Sea $h(a/s) = g(a)g(s)^{-1}$. Entonces si $h$ está bien definido será claramente un homomorfismo de anillos. Suponga que $a/s = a'/s'$; entonces $as' = a's$ y de aquí
  \[
    g(a)g(s') = g(a')g(s);
  \]
  ahora $g(s)$ y $g(s')$ son unidades en $B$ y por lo tanto $g(a)g(s)^{-1} = g(a')g(s')^{-1}$.
\end{proof}

\begin{example}~
\begin{enumerate}
  \item Sea $t$ un elemento distinto de cero de $A$; entonces
    \[
      S_t = \{1,t,t^2,\ldots\}
    \]
  es un subconjunto multiplicativo de $A$ y escribimos (algunas veces) $A_t$ en lugar de $S^{-1}A$. Por ejemplo si $d$ es un entero distinto de cero,
  \[
    \Z_d = \set{ \frac{a}{d^{n}} \in \Q : a \in \Z,  n \geq 0 }.
  \]
  Consiste de aquellos elementos de $\Q$ cuyo denominador es alguna potencia de $d$.

  \item Si $P$ es un ideal primo, entonces $S_P = A \sm P$ es un conjunto multiplicativo (si ni $a$ ni $b$ pertenecen a $P$, entonces $ab$ no pertenece  a $P$). Escribimos $A_P$ en su lugar $S^{-1}_P A.$ Por ejemplo,
  \[
    \Z_{(p)} = \set{\frac{m}{n} \in \Q : n \ \text{no es divisible por}\ p}.
  \]
\end{enumerate}
\end{example}

\begin{proposition}
  Sea $A$ un dominio entero y sea $S$ un subconjunto multiplicativo de $A$. Si $I$ es un ideal de $A$, entonces $I^{e} = S^{-1} I = S^{-1}A$ si y solo si $I \cap S \neq \emptyset$. La función
  \[
    P \mapsto S^{-1}P = \{\frac{a}{s} : a \in P, s \in S\}
  \]
  es una biyección del conjunto de ideales primos en $A$ tales que $P \cap S = \emptyset$ al conjunto de ideales primos en $S^{-1}A$; la función inversa es $Q \mapsto Q \cap A$.
\end{proposition}
\begin{proof}
  El ideal extendido $I^e = S^{-1}I$ por que cualquier $y \in I^e$ es de la forma $\sum a_i/s_i$, donde $a_i \in I$ y $s_i \in S$; lleve esta fracción a un denominador común. La afirmación $S^{-1} I = S^{-1}A$ si y solo si $I \cap S = \emptyset$ es trivial. Es fácil ver que si:
  \begin{itemize}
    \item $P$ es un ideal primo disjunto de $S$ implica que $S^{-1}P$ es un ideal primo.
    \item $Q$ es un ideal primo en $S^{-1}A$ implica que $Q \cap A$ es un ideal primo disjunto de $S$.
  \end{itemize}

  Por lo tanto lo único que tenemos que mostrar es que las dos funciones son inversas, i.e., 
  \[
    (S^{-1}P) \cap A = P
      \Eqand
    S^{-1}(Q \cap A) = Q.
  \]

  Para la primera igualdad. Claramente $(S^{-1}P) \cap A \supseteq P$. Para la otra inclusión, sea $a/s \in (S^{-1}P) \cap A$, $a\in P$, $s \in S$. Considere la ecuación $(a/s)s = a \in P$. Ya que ambos $a/s$ y $s$ están en $A$, entonces $a/s$ o $s$ está en $P$ (por que es primo); pero $s \notin P$ por hipótesis y por lo tanto $a/s \in P$.

  Para la segunda igualdad. Claramente $S^A{-1}(Q \cap A) \subseteq Q$ por que $Q \cap A .\subseteq Q$ y $Q$ es un ideal en $S^{-1}A$. Para la otra inclusión, sea $b \in Q$. Podemos escribirlo como $b = a/s$ con $a \in A$, $s \in S$. Entonces $a = s(a/s) \in Q \cap A$ y por lo tanto $a/s \in S^{-1}(Q \cap A)$.
\end{proof}

\begin{example}
  Si $P$ es un ideal primo de $A$, $A_P$ es un anillo local (por que $P$ contiene cada ideal primo disjunto de $S_P$). Listamos los ideales primos de algunos anillos:
  \begin{itemize}
    \item $\Z$: $(2)$, $(3)$, $(5)$, $(7)$, $(11)$, \dots, $(0)$;
    \item $\Z_2$: $(3)$, $(5)$, $(7)$, $(11)$, \dots, $(0)$;
    \item $\Z_(2)$: $(2)$, $(0)$;
    \item $\Z_{42}$: $(5)$, $(11)$, $(13)$, \dots, $(0)$;
    \item $\Z/(42)$: $(2)$, $(3)$, $(7)$.
  \end{itemize}

  Note que en general, si $t$ es un elemento distinto de cero de un dominio entero, 
  \begin{itemize}
    \item $\{\text{ideales primos de $A_t$}\} \leftrightarrow \{\text{ideales primos de $A$ que no contienen a $t$}\}$,
    \item  $\{\text{ideales primos de $A/(t)$}\} \leftrightarrow \{\text{ideales primos de $A$ que contienen a $t$}\}$.
  \end{itemize}
\end{example}




\section{Módulos libres}

El concepto de independencia lineal se extiende a los módulos.

\begin{definition}
  Un subconjunto $S$ de un módulo $M$ es \emph{linealmente independiente} si para cualesquiera $v_1,\ldots,v_n \in S$,
    \[
      a_1 v_1 + a_2 v_2 + \cdots + a_n v_n = 0 \implies a_1 = a_2 = \cdots = a_n = 0.
    \]
  Si un conjunto $S$ no es linealmente independiente, decimos que es \emph{linealmente dependiente}.
\end{definition}

En un espacio vectorial, el conjunto $S = \{v\}$, consistiendo de un solo vector $v$ distinto de cero, es linealmente independiente. Sin embargo, en un módulo, esto no es necesariamente cierto.

\begin{example}
  El grupo abeliano $\Z/(n) = \{0,1,\ldots,n-1\}$ es un $\Z$-módulo; sin embargo, ya que $na = 0$ para todo $a \in \Z/(n)$, ningún conjunto unitario $\{a\}$ es linealmente independiente.
\end{example}

También, en un espacio vectorial, un conjunto $S$ de vectores es linealmente dependiente si y solo si algún vector en $S$ es una combinación lineal de los otros vectores en $S$. Para módulos arbitrarios, esto no es verdad. El problema es que
\[
  a_1v_1 + a_2v_2 + \cdots + a_nv_n = 0
\]
y digamos, $a_1 \neq 0$ implica que 
\[
  a_1v_1 = -a_2v_2 - \cdots - a_nv_n
\]
pero en general, no podemos dividir ambos lados por $a_1$.

\begin{definition}
  Sea $M$ un $A$-módulo. Un subconjunto $B$ de $M$ es una \emph{base} si $B$ es linealmente independiente y genera $M$.
\end{definition}

\begin{theorem}
  Un subconjunto $B$ de un módulo $M$ es una base si y solo si, para cada $v \in M$, existe un conjunto único de escalares $a_1,\ldots,a_n$ para los cuales
  \[
    v = a_1v_1 + a_2v_2 + \cdots + a_nv_n \tag*{\qed}
  \]
\end{theorem}

En un espacio vectorial, un conjunto de vectores es una base si y solo si es un conjunto de generadores mínimo o, equivalentemente, un conjunto linealmente independiente máximo. Para módulos, lo más que se puede hacer es lo siguiente:

\begin{theorem}\label{theo:1.7.5}%
  Sea $B$ una base para un $A$-módulo $M$. Entonces
  \begin{enumerate}
    \item $B$ es un conjunto generador mínimo.
    \item $B$ es un conjunto linealmente independiente máximo. \qed
  \end{enumerate}
\end{theorem}

El $\Z$-módulo $\Z/(n)$ es un ejemplo de un módulo que no tiene base, ya que no tiene conjuntos linealmente independientes, excepto por el conjunto vacío. Sin embargo, ya que todo el módulo es un conjunto generador, un conjunto generador mínimo no necesariamente tiene una base.

Introducimos ahora la suma directa y el producto directo de $A$-módulos. Si $(M_i)_{i \in I}$ es una familia cualquiera de $A$-módulos, se puede definir su \emph{suma directa} $\bigoplus_{i \in I} M_i$; sus elementos son familias $(x_i)_{i \in I}$ tales que $x_i \in M_i$ para cada $i \in I$ y casi todas las $x_i$ son nulas. Si omitimos la restricción del número de $x_i$ no nulas se tiene el \emph{producto directo} $\prod_{i \in I} M_i$. La suma y el producto directo son iguales si el conjunto de índices es finito.

El hecho de que no todos los módulos tienen una base nos condice a la definición siguiente.

\begin{definition}
  Un $A$-módulo $M$ es \emph{libre} si tiene una base. Si $B$ es una base para $M$, decimos que $M$ es libre sobre $B$.
\end{definition}

El ejemplo siguiente muestra que aún los módulos libres no son muy parecidos a los espacios vectoriales. Es un ejemplo de un módulo libre que tiene un submódulo que no es libre.

\begin{example}
  El conjunto $\Z \times \Z$ es un módulo libre sobre sí mismo, con base $\{(1,1)\}$. Para ver esto observe que $(1,1)$ es linealmente independiente, ya que
  \[
    (n,m)(1,1) = (0,0) \implies (n,m) = 0.
  \]

  También, $(1,1)$ genera $\Z \times \Z$ ya que $(n,m) = (n,m)(1,1)$. Pero el submódulo $S = \Z \times \{0\}$ no es libre, ya que no tiene elementos linealmente independientes y de aquí no tiene base. Esto se sigue del hecho de que si $(n,0) \neq (0,0)$, entonces, por ejemplo $(0,1)(n,0) = (0,0)$ y así $\{(n,0)\}$ no es linealmente independiente.
\end{example}

Una conexión entre módulos libres y sumas directas es dada por la proposición siguiente.

\begin{proposition}~
  \begin{enumerate}
    \item Sea $(M_i)_{i \in I}$ una familia de $A$-módulos con $M_i = A$ para toda $i \in I$. Entonces $\bigoplus_{i\in I} M_i$ es un $A$-módulo libre, con (usando $I \neq \emptyset$) base $(e_i)_{i \in I}$, donde para cada $i \in I$, el elemento $e_i \in \bigoplus_{i\in I} M_i$ tiene su componente en $M_i$ igual a $1$ y todos sus otros componentes iguales a cero.
    
    \item Sea $M$ un $A$-módulo. Entonces $M$ es libre si y solo si $M$ es isomorfo a un $A$-módulo del tipo descrito en $(a)$. Esto es, si $M$ es isomorfo a una suma directa de copias de $A$.
    
    De hecho, si $M$ tiene una base $(e_i)_{i \in I}$, entonces $M \cong \bigoplus_{i\in I} M_i$, donde $M_i = A$ para toda $i \in I$.
  \end{enumerate}
\end{proposition}
\begin{proof}
  $(a)$ es claro. Para la ida de $(b)$. Sea $M$ un $A$-módulo libre con base $\{e_i\}_{i \in I}$. Para cada $i \in I$, sea $M_i =A$ y defina
    \[
      f\colon \bigoplus_{i \in I} M_i \to M
    \]
    como $f((a_i)_{i\in I}) = \sum_{i\in I} a_i e_i$. Entonce $f$ es un $A$-homomorfismo. Ya que $\{e_i : i \in I\}$ es un conjunto generador de $M$, $f$ es suprayectivo. Como $\{e_i\}_{i\in I}$ es una base, $f$ es inyectivo.

    Para la vuelta de $(b)$. Si $M'$ y $M''$ son $A$-módulos isomorfos, entonces $M'$ es libre si y solo si $M''$ es libre y de aquí, esta parte se sigue de $(a)$.
\end{proof}

De la misma manera que en la teoría de espacio vectoriales, las bases permiten una descripción fácil de las transformaciones lineales, una base para un $A$-módulo libre $F$ permite una descripción fácil de los $A$-homomorfismos de $F$ a otros $A$-módulos.

\begin{proposition}\label{prop:1.7.9}%
  Un $A$-módulo $F$ es libre si y solo si existe un conjunto $X$ y una función $i\colon X\to F$ con la propiedad universal siguiente: dado cualquier $A$-módulo $M$ y una función $f\colon X \to M$, existe un único $A$-homomorfismo $\bar f \colon F \to M$ tal que $\bar f \circ i = f$. En otras palabras, $F$ es un objeto libre en la categoría de $A$-módulos.
\end{proposition}
\begin{proof}
  $(\Rightarrow)$ Sea $X = B$ una base de $F$ e $i\colon X \to F$ la inclusión. Suponga que nos dan la función $f \colon X \to M$. Si $u\in F$, entonces $u = \sum_{i=1}^n a_ix_i$, $a_I \in A$, $x_i \in X$, ya que $X$ es una base de $F$. Ya que esta expresión es única, el mapeo $\bar f \colon F \to M$ dado por
  \[
    \bar f (u) = \bar f \paren{\sum_{i=1}^n a_ix_i} = \sum_{i=1}^n a_if(x_i)
  \]
  está bien definido, $\bar f \circ i = f$ y $\bar f$ es un homomorfismo de $A$-módulos. Ya que $X$ genera $F$, cualquier $A$-homomorfismo está determinado de manera única por su acción sobre $X$. De este modo, si $g \colon F \to M$ es tal que $g\circ i = f$, entonces $g(x) = g(i(x)) = f(x) = \bar f(x)$, de donde $g = \bar f$ y $\bar f$ es único.

  $(\Leftarrow)$ Considere el módulo libre $A^X \coloneqq \bigoplus_{x \in X} M_x$, donde $M_x = A$ para todo $x \in X$. Sea $f\colon X \to A^X$ la función dada por $f(x) = e_x$, donde $e_x$, donde $e_x$ es el vector básico canónico de $A^X$. Entonces, ya que $F$ satisface la propiedad universal, existe un homomorfismo de $A$-módulos $\bar f \colon F \to A^X$ tal que el diagrama siguiente es conmutativo,
  \[
    \begin{tikzcd}
      X \arrow[r, "i"] \arrow[rd, "f"'] & F \arrow[d, "\bar f"] \\
                                        & A^X                  
      \end{tikzcd}
  \]
i.e., $\bar f \circ i = f$. Sea $Y = \{e_x : x \in X\}$ la base canónica de $A^X$. Ya que $A^X$ es libre, satisface la propiedad universal y por lo tanto dada la función $f \colon Y \to F$, $g(e_x) = i(x)$ existe un $A$\nobreakdash-homomorfismo $\bar g \colon A^X \to F$ tal que el triangulo superior del diagrama siguiente es conmutativo:
\[
  \begin{tikzcd}
                                                    & A^X \arrow[d, "\bar g"] \\
  Y \arrow[ru, "j"] \arrow[rd, "j"'] \arrow[r, "g"] & F \arrow[d, "\bar f"]   \\
                                                    & A^X                    
  \end{tikzcd}
\]
donde $j(e_x) = e_x$ es la inclusión de $Y$ en $A^X$. Note que el triángulo inferior también es conmutativo ya que $(\bar f \circ g)(e_x) = (\bar f \circ i)(x) = f(x) = e_x = j(e_x)$. De aquí, el $A$-homomorfismo $\bar f \circ \bar g$ satisface $\bar f \circ \bar g \circ j = \bar f \circ g = j$. Pero el homomorfismo identidad $1_{A^X} \colon A^X \to A^X$ satisface lo mismo, así que por unicidad $\bar f \circ \bar g = 1_{A^X}$. Similarmente $(\bar g \circ \bar f \circ i)(x) = (\bar g \circ f)(x) = \bar g(e_x) = (\bar g \circ j)(e_x) = g(e_x) = i(x)$. De nuevo, por unicidad $\bar g \circ \bar f = 1_F$. Esto muestra que $\bar f$ y $\bar g$ son $A$-isomorfismos y por lo tanto $F$ al ser isomorfismo a un módulo libre es un módulo libre y una base de $F$ es $\bar g(Y) = i(X)$.
\end{proof}

Si $X$ es cualquier conjunto no vacío, podemos construir un $A$-módulo libre $F$ sobre el conjunto $X$. Sea $F$ el conjunto de expresiones formales $\sum_{x\in X} a_x x$ donde $a_x \in A$ y únicamente un número finito de los $a_x$ son distintos de cero. Entonces si definimos
\begin{align*}
  \sum a_x x +  \sum b_x x &= \sum (a_x + b_x) x, \\
  a \sum a_x x &= \sum (aa_x) x,
\end{align*}
$F$ es un $A$-módulo. Si identificamos cada $x \in x$ con la expresión $\sum a_x x \in F$ con $a_x = 1$ y $a_y = 0$ para $y \neq x$, entonces $X$ es una base de $F$.

Un módulo arbitrario es una imagen homomorfa de un módulo libre.

\begin{proposition}
  Sea $M$ un $A$-módulo. Existe un $A$-módulo libre $F$ y un $A$-epimorfismo $f\colon F \to M$. Si $M$ es finitamente generado por $n$ elementos, entonces podemos escoger $F$ de mono tal que tenga una base de $n$ elementos.
\end{proposition}
\begin{proof}
  Sea $X$ un conjunto que genera $M$ y $F$ un $A$-módulo con base $X$. Defina $f\colon F \to M$ usando la proposición \ref{prop:1.7.9}. Claramente $f$ es suprayectiva.
\end{proof}

Nuestro siguiente resultado en esta sección introduce el concepto de dimensión de un módulo libre.

\begin{definition}
  Sea $M$ un $A$-módulo libre. La \emph{dimensión} (o \emph{rango}) de $M$ es la cardinalidad de cualquier base de $M$.
\end{definition}

La dimensión está bien definida por el teorema siguiente.

\begin{theorem}
  Sea $M$ un $A$-módulo libre. Entonces dis bases cualesquiera de $M$ tienen la misma cardinalidad.
\end{theorem}
\begin{proof}
  Sea $I$ un ideal maximal de $A$. Entonces $A/I$ es un campo. Entonces $M/IM$ es un espacio vectorial sobre $A/I$ con multiplicación escalar definida como
  \[
    (a+I)(m+IM) = am + IM.
  \]

  Sea $B$ una base para $M$ sobre $A$. Si $b_i$ y $b_j \in B$ entonces $b_i + IM \neq b_j + IM$ por que si
  \[
    b_i + IM = b_j + IM
  \]
  entonces $b_i-b_j \in IM$ y por lo tanto $b_i - b_j = \sum_{i=1}^n a_i x_i$ con $a_i \in I$ y $x_i \in M$. Pero cada $x_i$ es una combinación lineal de los vectores de la base $B$. Supongamos que el coeficiente de $b_i$ en $x_k$ es $c_k$, $k=1,\ldots,n$. Igualando los coeficientes de ambos lados
  \[
    1 = a_1 c_1 + \cdots + a_n c_n.
  \]
  Ya que la suma está en $I$, $1 \in I$, lo cual es una contradicción. Entonces
  \[
    B' = \{b+IM : b \in B\}
  \]
  es una base de $M/IM$ sobre $A/I$. Claramente $B'$ genera $M/IM$ pues $B$ genera $M$. Para ver que son linealmente independientes,
  \[
    \sum (a_j + I) (b_j + IM) = 0 \implies \sum (a_jb_j + IM) = 0 \implies \sum a_j b_j \in IM.
  \]
  Aplicando definición y simplificando tenemos que $\sum a_j b_j = \sum c_i b_i$ para algunos $c_i \in I$. Igualando los coeficientes de los $b_j$ obtenemos que $a_j \in I$. De este modo $\abs{B} = \dim (M/IM)$ es independiente de la base $B$ escogida.
\end{proof}

\begin{definition}\label{def:1.7.13}%
  Para cualquier dominio entero $A$ el \emph{rango} de un $A$-módulo $M$ es el número máximo de elementos $A$-linealmente independientes de $M$.
\end{definition}

Un $A$-módulo de rango $n$ no necesariamente tiene una base, i.e., no es necesariamente un módulo libre.

Sea $A$ un dominio entero y $M$ un módulo libre de dimensión $n$. Entonces, por el teorema \ref{theo:1.7.5}, $n+1$ elementos cualesquiera de $M$ son $A$-linealmente dependientes. De aquí la dimensión de un módulo libre coincide con el rango definido en la definición \ref{def:1.7.13}. También se sigue que el rango de un submódulo de un módulo libre $M$ está acotado por al dimensión de $M$.

\begin{theorem}
  Sea $M$ un módulo libre de dimensión $n$ sobre un dominio de ideales principales $A$ y $N$ un submódulo de $M$. Entonces
  \begin{enumerate}
    \item $N$ es libre de rango $m$, $m \leq n$.
    \item Existe una base $y_1,\ldots,y_n$ de $M$ tal que $a_1y_1,\ldots,a_m y_n$ es una base de $N$ donde $a_1,\ldots,a_m$ son elementos de $A$ distintos de cero con las relaciones de divisibilidad siguientes
      \[
        a_1 \mid a_2 \mid \cdots \mid a_m.
      \]
  \end{enumerate}
\end{theorem}
\begin{proof}
  El teorema es trivial para $N = \{0\}$, así que supongamos que $N \neq \{0\}$. Para cada homomorfismo de $A$-módulos $\varphi$ de $M$ a $A$, la imagen $\varphi(N)$ de $N$ es un submódulo de $A$, i.e., un ideal en $A$. Ya que $A$ es un dominio de ideales principales, este ideal debe ser principal, digamos $\varphi(N) = (a_\varphi)$ para algún $a_\varphi \in A$. Sea
  \[
    \Sigma = \{ (a_\varphi) : \varphi \in \Hom_A(M,A) \}.
  \]

  La colección $\Sigma \neq \emptyset$ pues si $\varphi = 0$, $(0) \in \Sigma$. Ya que $A$ es noetheriano, $\Sigma$ tiene un elemento maximal. Sea $a_1 = \nu(y)$ este elemento maximal.

  Sea $x_1,\ldots,x_n$ una base cualquiera de $M$ y sea $\pi_i \in \Hom_A(M,A)$ la proyección natural en la coordenada $i$-ésima con respecto a esta base. Ya que $N \neq \{0\}$, existe $i$ tal que $\pi_i(N) \neq 0$, que en particular muestra que $\Sigma = \{ (0)\}$.

  Ya que $(a_1)$ es un elemento maximal de $\Sigma$ se sigue que $a_1 \neq 0$.

  Sea $\varphi \in \Hom_A(M,A)$ y $(d) = (a_1, \varphi(y))$, entonces $d$ es un divisor de ambos $a_1$, $\varphi(y)$ en $A$ y $d = r_1 a_1 + r_2 \varphi(y)$ para algunos $r_1, r_2 \in A$. Considere el homomorfismo $\psi = r_1 \nu + r_2 \varphi$ de $M$ a $A$. Entonces $\psi(y) = (r_1 \nu + r_2 \varphi)(y) = r_1a_1 + r_2 \varphi(y) = d$ así que $d \in \psi(N)$, de aquí $(d) \subseteq \psi(N)$. Ya que $d$ divide a $a_1$, $(a_1) \subseteq (d) \subseteq \psi(N)$ y por la maximalidad de $(a_1)$, $(a_1) = (d) = \psi(N)$. En particular, $(a_1) = (d)$ muestra que $a_1 \mid \varphi(y)$ ya que $d \mid \varphi(y)$.

  Aplicando esto a la proyección $\pi_1$ vemos que $a_1$ divide $\pi_i(y)$ para toda $i$. Escriba $\pi_i(y) = a_1 b_i$ para algún $b_i \in A$, $1 \leq i \leq n$ y defina
  \[
    y_1 = \sum_{i=1}^n b_i x_i.
  \]

  Notemos que $a_1 y_1 = \sum_{i=1}^n a_1 b_i x_i = \sum_{i=1}^n \pi_i (y) x_i = y$. Ya que $a_1 = \nu(y) = \nu(a_1 y_1) = a_1 \nu(y_1)$ y $a_1 \neq 0$, $\nu(y_1) = 1$.

  Ahora verificamos que este elemento $y_1$ puede ser tomado como un elemento en una base para $M$ y que $a_1 y_1$ puede ser tomado como un elemento en una base para $M$ y que $a_1 y_1$ puede ser tomado como un elemento en una base para $N$, a saber, que tenemos que
  \begin{enumerate}[label=\arabic*.]
    \item $M = Ay_1 \oplus \ker \nu$,
    \item $N = Aa_1y_1 \oplus (N \cap \ker \nu)$.
  \end{enumerate}
  Para ver 1, sea $x$ un elemento arbitrario en $M$ y escriba $x = \nu(x)y_1 + (x-\nu(x)y_1)$. Ya que
  \begin{align*}
    \nu(x-\nu(x)y_1) &= \nu(x) - \nu(x)\nu(y_1), \\
      &= \nu(x) - \nu(x)\cdot 1, \\
      &= 0,
  \end{align*}
  vemos que $x - \nu(x)y_1$ es un elemento del núcleo del $\nu$. Esto muestra que $x$ puede ser escrito como la suma de un elemento en $Ay_1$ y un elemento en el núcleo de $\nu$, así que $M = Ay_1 + \ker \nu$.

  Suponga que ahora que $ry_1$ es también un elemento en el núcleo de $\nu$. Entonces $0 = \nu(ry_1) = r\nu(y_1) = r$ que muestra que este elemento es $0$.

  Para 2, observe que $\nu(x')$ es divisible por $a_1$ para cada $x' \in \N$ por la definición de $a_1$ como un generador de $\nu(N)$. Si escribimos $\nu(x') = ba_1$ donde $b \in A$ entonces
  \begin{align*}
    x &= \nu(x')y_1 + (x' - \nu(x')y_1) \\
      &= ba_1y_1 + (x' - ba_1y_1)
  \end{align*}
  donde el segundo sumando está en el núcleo de $\nu$ y es un elemento de $N$. Esto muestra que $N = Aa_1y_1 + (N \cap \ker \nu)$. El hecho de que la suma en 2 es directa es un caso especial de la suma directa en 1.

  Probamos $a)$ por inducción sobre $m$. Si $m=0$, entonces 
  \[
    \Tor(N) = \{x \in N : rx = 0 \ \text{para algún}\ r \neq 0 \in A\}
  \]
  es un submódulo de $N$ (llamado el \emph{submódulo de torsión} de $N$) que es igual a $N$. Ya que $M$ es libre, $\Tor(M) = 0$ y de aquí $N=0$. Así que $N$ es libre de dimensión $0$.

  Supongamos entonces que $m>0$. Ya que la suma en 2 es directa vemos que el número máximo de elementos de $N \cap \ker \nu$ que son linealmente independientes es $(m-1)$ (ejercicio 4). Por inducción, $N \cap \nu$ es libre de dimensión $m-1$. De nuevo, ya que la suma en 2 es directa vemos que adjuntado $a_1 y_1$ a cualquier base de $N \cap \ker \nu$  da una base de $N$, así que $N$ es libre de dimensión $m$, lo que prueba $a)$.

  Finalmente, probamos $b)$ por inducción sobre $n$, la dimensión de $M$. Aplicando $a)$ al submódulo $\ker\nu$ vemos que este submódulo es libre y porque la suma en 1 es directa es libre de dimensión $n-1$. Por inducción, existe una base $y_1, y_3,\ldots,y_n$ de $\ker \nu$ tal que $a_2y_2,a_3y_3\ldots,a_my_m$ es una base de $N \cap \ker \nu$ para algunos $a_2,a_3,\ldots,a_m$ de $A$ con $a_2 \mid a_3 \mid \cdots \mid a_m$. Ya que las sumas en 1 y 2 son directas, $y_1,\ldots,y_n$ es una base de $M$ y $a_1y_1,\ldots,a_my_m$ es una base de $N$. Sólo resta probar que $a_1 \mid a_2$.

  Defina un homomorfismo $\varphi$ de $M$ a $A$ definiendo $\varphi(y_1) = \varphi(y_2) = 1$ y $\varphi(y_i) = 0$ para toda $i>0$ sobre la base de $M$. Entonces para este homomorfismo $\varphi$ tenemos que $a_1 = \varphi(a_1y_1)$ así que $a_1 \in \varphi(N)$ y de aquí, $(a_1) \subseteq \varphi(N)$. Por la maximalidad de $(a_1)$ en $\Sigma$ se sigue que $(a_1) = \varphi(N)$. Ya que $a_2 = \varphi(a_2y_2) \in \varphi(N)$, $a_2 \in (a_1)$, i.e., $a_1 \mid a_2$.
\end{proof}



\ExerciseSection

\begin{exerciseList}
  \item Pruebe el teorema 1.7.5.
  
  \item Sea $M$ un módulo sobre el dominio entero $A$.
    \begin{enumerate}
      \item Suponga que $x$ es un elemento de torsión de $M$ distinto de cero. Muestre que $x$ y $0$ son ``linealmente dependientes''. Concluya que el rango de $\Tor(M) = \{x \in M : rx = 0 \ \text{para algún}\ r \neq 0 \in A\}$ es cero, así que en particular cualquier $A$-módulo de torsión, i.e., cualquier $A$-módulo $M$ con $\Tor(M) = M$ tiene rango $0$.
      \item Muestre que el rango de $M$ es el mismo que el rango del módulo cociente (libre de torsión) $M/\Tor(M)$.
    \end{enumerate}
    
    \item Sea $M$ un módulo sobre el dominio entero $A$.
    \begin{enumerate}
      \item Suponga que $M$ tiene rango $n$ y que $x_1,x_2,\ldots,x_n$ es un conjunto maximal de elementos linealmente independientes de $M$. Sea $N = Ax_1 + \cdots + Ax_n$ el submódulo generado por $x_1,x_2,\ldots,x_n$. Pruebe que $N$ es isomorfo a $A^n$ y que el cociente $M/N$ es un $A$-módulo de torsión (equivalentemente, los elementos $x_1,x_2,\ldots,x_n$ son linealmente independientes y para cualquier $y \in M$ existe un elemento $r \neq 0 \in A$ tal que $ry$ puede ser escrito como una combinación lineal $r_1x_1 + \cdots + r_nx_n$ de los $x_i$).
      \item Recíprocamente, pruebe que si $M$ contiene un submódulo $N$ que es libre de rango $n$ (i.e., $N \cong A^n$) tal que el cociente $M/N$ es un $A$-módulo de torsión entonces $M$ tiene rango $n$. (Sean $y_1,y_2,\ldots,y_{n+1}$, $n+1$ elementos cualesquiera de $M$. Use el hecho de que $M/N$ es de torsión para escribir $r_iy_i$ como una combinación lineal de una base para $N$ para algunos elementos de $A$ distintos de cero $r_1,\cdots,r_{n+1}$. Muestre que los $r_iy_i$ son linealmente dependientes y de aquí, también los $y_i$).
      \item Sea $A$ un dominio entero y sean $M$ y $N$, $A$-módulos de rangos $m$ y $n$, respectivamente. Pruebe que el rango de $M \oplus N$ es $m+n$. (Use el ejercicio anterior).
    \end{enumerate}
\end{exerciseList}